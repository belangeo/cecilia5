%!TEX TS-program = /usr/texbin/pdflatex
\documentclass[11pt,francais]{book}
\usepackage{amssymb,amsmath,epsfig}
\usepackage{graphics}
\usepackage[utf8]{inputenc}
\usepackage[francais]{babel}
\usepackage{listings}
\usepackage{color}
\usepackage[pdfborder=0]{hyperref}
\hypersetup{
urlcolor=blue,
linkcolor=blue,
colorlinks=true, 
linkbordercolor={1 1 1}, % set to white
citebordercolor={1 1 1} % set to white
}

\oddsidemargin 0.2in
\evensidemargin 0in
\marginparwidth 40pt
\marginparsep 10pt
\topmargin 0pt
\headsep .5in
\textheight 8.35in \textwidth 6.3in

\definecolor{green}{rgb}{0,0.7,0}
\definecolor{dkgreen}{rgb}{0,0.5,0}
\definecolor{verylightgray}{rgb}{0.97,0.97,0.97}
\definecolor{lightgray}{rgb}{0.95,0.95,0.95}
\definecolor{gray}{rgb}{0.6,0.6,0.6}
\definecolor{dkgray}{rgb}{0.4,0.4,0.4}

%
% Il faut installer le package lstlisting pour obtenir le formattage du code source
% http://www.ctan.org/tex-archive/macros/latex/contrib/listings/
%

\lstdefinestyle{Python}{ %
  language=Python,                % the language of the code
  basicstyle=\footnotesize,       % the size of the fonts that are used for the code
  numbers=left,                   % where to put the line-numbers
  numberstyle=\tiny\color{gray},  % the style that is used for the line-numbers
  stepnumber=1,                   % the step between two line-numbers. If it's 1, each line will be numbered
  numbersep=10pt,                  % how far the line-numbers are from the code
  backgroundcolor=\color{lightgray},  % choose the background color. You must add \usepackage{color}
  showspaces=false,               % show spaces adding particular underscores
  showstringspaces=false,         % underline spaces within strings
  showtabs=false,                 % show tabs within strings adding particular underscores
  frame=false,                    % adds a frame around the code
  rulecolor=\color{lightgray},    % if not set, the frame-color may be changed on line-breaks within not-black text 
  tabsize=4,                      % sets default tabsize to 2 spaces
  captionpos=b,                   % sets the caption-position to bottom
  breaklines=true,                % sets automatic line breaking
  breakatwhitespace=false,        % sets if automatic breaks should only happen at whitespace
  title=\lstname,                 % show the filename of files included with \lstinputlisting;
                                  % also try caption instead of title
  keywordstyle=\color{blue},      % keyword style
  commentstyle=\color{dkgray},      % comment style
  stringstyle=\color{dkgreen},    % string literal style
  escapeinside={\%*}{*)},         % if you want to add a comment within your code
  morekeywords={Abs, Atan2, Ceil, Cos, Floor, Log, Log10, Log2, Pow, Round, Sin, Sqrt, Tan, CarToPol, FFT, FrameAccum, FrameDelta, IFFT, PolToCar, Vectral, MatrixMorph, MatrixPointer, MatrixRec, MatrixRecLoop, OscDataReceive, OscDataSend, OscListReceive, OscReceive, OscSend, Bendin, CtlScan, MidiAdsr, MidiDelAdsr, Midictl, Notein, Programin, Touchin, CallAfter, Pattern, Score, AToDB, Between, CentsToTranspo, Clean_objects, Compare, ControlRead, ControlRec, DBToA, Denorm, Interp, MToF, MToT, NoteinRead, NoteinRec, Print, Record, SampHold, Scale, Snap, TranspoToCents, Dummy, InputFader, Mix, VarPort, Follower, Follower2, ZCross, SfMarkerLooper, SfMarkerShuffler, SfPlayer, Choice, RandDur, RandInt, Randh, Randi, Urn, Xnoise, XnoiseDur, XnoiseMidi, Blit, BrownNoise, CrossFM, FM, Input, LFO, Lorenz, Noise, Phasor, PinkNoise, Rossler, Sine, SineLoop, Adsr, Expseg, Fader, Linseg, Sig, SigTo, Allpass, Allpass2, BandSplit, Biquad, Biquada, Biquadx, DCBlock, EQ, FourBand, Hilbert, IRAverage, IRFM, IRPulse, IRWinSinc, Phaser, Port, Tone, Clip, Compress, Degrade, Gate, Mirror, Wrap, Granulator, Lookup, Looper, Osc, OscBank, OscLoop, Pointer, Pulsar, TableIndex, TableMorph, TableRead, TableRec, Beat, Change, Cloud, Count, Counter, Iter, Metro, NextTrig, Percent, Select, Seq, Thresh, Timer, Trig, TrigChoice, TrigEnv, TrigExpseg, TrigFunc, TrigLinseg, TrigRand, TrigRandInt, TrigTableRec, TrigXnoise, TrigXnoiseMidi, Mixer, Pan, SPan, Selector, Switch, VoiceManager, AllpassWG, Chorus, Convolve, Delay, Disto, Freeverb, FreqShift, Harmonizer, SDelay, Vocoder, WGVerb, Waveguide, SLMapDur, SLMapFreq, SLMapMul, SLMapPan, SLMapPhase, SLMapQ, class_args, convertStringToSysEncoding, distanceToSegment, downsamp, example, getVersion, linToCosCurve, midiToHz, midiToTranspo, pa_count_devices, pa_count_host_apis, pa_get_default_host_api, pa_get_default_input, pa_get_default_output, pa_get_input_devices, pa_get_output_devices, pa_list_devices, pa_list_host_apis, pm_count_devices, pm_get_default_input, pm_get_default_output, pm_get_input_devices, pm_get_output_devices, pm_list_devices, reducePoints, rescale, sampsToSec, savefile, savefileFromTable, secToSamps, serverBooted, serverCreated, sndinfo, upsamp, ChebyTable, CosTable, CurveTable, DataTable, ExpTable, HannTable, HarmTable, LinTable, NewTable, ParaTable, SawTable, SincTable, SndTable, SquareTable, WinTable, NewMatrix, PyoObject, PyoTableObject, PyoMatrixObject, Server},% if you want to add more keywords to the set
  morestring=[b][\color{green}]{"""},
  morecomment=[l][\color{dkgray}]{\#},
  morecomment=[l][\color{gray}]{\#\#\#},
}

\newcommand\IMGPATH{images/}

\newcommand{\insertImage}[3]{
    \medskip
    \begin{figure}[htbp]
        \begin{center}
            \includegraphics[width=#1in]{\IMGPATH #2}
        \end{center}
        \caption{#3}
    \end{figure}
}

\newcommand{\insertImageLeft}[3]{
    \begin{table}[ht]
        \begin{minipage}[b]{0.5\linewidth}
            \includegraphics[width=#1in]{\IMGPATH #2}
        \end{minipage}
        \hspace{0.1in}
        \begin{minipage}[b]{0.5\linewidth}
            {#3}
        \end{minipage}
    \end{table}
}

\newcommand{\insertImageRight}[3]{
    \begin{table}[ht]
        \begin{minipage}[b]{0.5\linewidth}
            {#3}
        \end{minipage}
        \hspace{0.1in}
        \begin{minipage}[b]{0.5\linewidth}
            \includegraphics[width=#1in]{\IMGPATH #2}
        \end{minipage}
    \end{table}
}

\begin{document}

\title{Cecilia5 - Manuel d'utilisation}
\author{iACT - 2012}
\maketitle
\tableofcontents

\chapter{Template}
\section{Hiérarchie}

\subsection{Chapitres}

Chaque chapitre (chacun couvrant une section de l'interface ou une section de tutoriel) doit être écrit dans un fichier séparé dans le dossier "chapitres".
Les chapitres sont inclus dans le fichier principal, "Cecilia5\_manuel.tex". Ex.:

\begin{verbatim}
\chapter{Template}
\section{Hiérarchie}

\subsection{Chapitres}

Chaque chapitre (chacun couvrant une section de l'interface ou une section de tutoriel) doit être écrit dans un fichier séparé dans le dossier "chapitres".
Les chapitres sont inclus dans le fichier principal, "Cecilia5\_manuel.tex". Ex.:

\begin{verbatim}
\chapter{Template}
\section{Hiérarchie}

\subsection{Chapitres}

Chaque chapitre (chacun couvrant une section de l'interface ou une section de tutoriel) doit être écrit dans un fichier séparé dans le dossier "chapitres".
Les chapitres sont inclus dans le fichier principal, "Cecilia5\_manuel.tex". Ex.:

\begin{verbatim}
\chapter{Template}
\input{chapitres/template}
\end{verbatim}

\noindent
Les tags :

\begin{verbatim}
\section{titre}
\subsection{titre}
\subsubsection{titre}
\end{verbatim}

\noindent
seront utilisés pour délimiter les sections et créer la table des matières.

\subsection{Images}

Les images, en format PDF, doivent résider dans un dossier, portant le nom du chapitre, à l'intérieur du dossier "images".
Une image simple, centrée, est insérée à l'aide de la commande "insertImage", qui prend 3 paramètres, soit:

\begin{enumerate}
    \item La taille de l'image est pouce.
    \item Le path du fichier, à partir du dossier "images".
    \item La légende à afficher.
\end{enumerate}

\noindent
Exemple d'image insérée à l'aide de la ligne suivante:

\begin{verbatim}
\insertImage{3}{template/preferences}{Exemple d'insertion d'image simple.}
\end{verbatim}

\insertImage{3}{template/preferences}{Exemple d'insertion d'image simple.}

\insertImageLeft{2.5}{template/postprocessing}{
La section Post-Processing permet d'appliquer des effets à la suite du traitement produit par le module courant.
\bigskip
\bigskip
\bigskip
\bigskip
\bigskip
}

\insertImageRight{2.5}{template/postprocessing}{
La section Post-Processing permet d'appliquer des effets à la suite du traitement produit par le module courant.
\bigskip
\bigskip
\bigskip
\bigskip
\bigskip
}

\end{verbatim}

\noindent
Les tags :

\begin{verbatim}
\section{titre}
\subsection{titre}
\subsubsection{titre}
\end{verbatim}

\noindent
seront utilisés pour délimiter les sections et créer la table des matières.

\subsection{Images}

Les images, en format PDF, doivent résider dans un dossier, portant le nom du chapitre, à l'intérieur du dossier "images".
Une image simple, centrée, est insérée à l'aide de la commande "insertImage", qui prend 3 paramètres, soit:

\begin{enumerate}
    \item La taille de l'image est pouce.
    \item Le path du fichier, à partir du dossier "images".
    \item La légende à afficher.
\end{enumerate}

\noindent
Exemple d'image insérée à l'aide de la ligne suivante:

\begin{verbatim}
\insertImage{3}{template/preferences}{Exemple d'insertion d'image simple.}
\end{verbatim}

\insertImage{3}{template/preferences}{Exemple d'insertion d'image simple.}

\insertImageLeft{2.5}{template/postprocessing}{
La section Post-Processing permet d'appliquer des effets à la suite du traitement produit par le module courant.
\bigskip
\bigskip
\bigskip
\bigskip
\bigskip
}

\insertImageRight{2.5}{template/postprocessing}{
La section Post-Processing permet d'appliquer des effets à la suite du traitement produit par le module courant.
\bigskip
\bigskip
\bigskip
\bigskip
\bigskip
}

\end{verbatim}

\noindent
Les tags :

\begin{verbatim}
\section{titre}
\subsection{titre}
\subsubsection{titre}
\end{verbatim}

\noindent
seront utilisés pour délimiter les sections et créer la table des matières.

\subsection{Images}

Les images, en format PDF, doivent résider dans un dossier, portant le nom du chapitre, à l'intérieur du dossier "images".
Une image simple, centrée, est insérée à l'aide de la commande "insertImage", qui prend 3 paramètres, soit:

\begin{enumerate}
    \item La taille de l'image est pouce.
    \item Le path du fichier, à partir du dossier "images".
    \item La légende à afficher.
\end{enumerate}

\noindent
Exemple d'image insérée à l'aide de la ligne suivante:

\begin{verbatim}
\insertImage{3}{template/preferences}{Exemple d'insertion d'image simple.}
\end{verbatim}

\insertImage{3}{template/preferences}{Exemple d'insertion d'image simple.}

\insertImageLeft{2.5}{template/postprocessing}{
La section Post-Processing permet d'appliquer des effets à la suite du traitement produit par le module courant.
\bigskip
\bigskip
\bigskip
\bigskip
\bigskip
}

\insertImageRight{2.5}{template/postprocessing}{
La section Post-Processing permet d'appliquer des effets à la suite du traitement produit par le module courant.
\bigskip
\bigskip
\bigskip
\bigskip
\bigskip
}


\end{document}
