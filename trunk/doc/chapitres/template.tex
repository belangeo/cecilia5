\section{Hiérarchie}

\subsection{Chapitres}

Chaque chapitre (chacun couvrant une section de l'interface ou une section de tutoriel) doit être écrit dans un fichier séparé dans le dossier "chapitres".
Les chapitres sont inclus dans le fichier principal, "Cecilia5\_manuel.tex". Ex.:

\begin{verbatim}
\chapter{Template}
\section{Hiérarchie}

\subsection{Chapitres}

Chaque chapitre (chacun couvrant une section de l'interface ou une section de tutoriel) doit être écrit dans un fichier séparé dans le dossier "chapitres".
Les chapitres sont inclus dans le fichier principal, "Cecilia5\_manuel.tex". Ex.:

\begin{verbatim}
\chapter{Template}
\section{Hiérarchie}

\subsection{Chapitres}

Chaque chapitre (chacun couvrant une section de l'interface ou une section de tutoriel) doit être écrit dans un fichier séparé dans le dossier "chapitres".
Les chapitres sont inclus dans le fichier principal, "Cecilia5\_manuel.tex". Ex.:

\begin{verbatim}
\chapter{Template}
\section{Hiérarchie}

\subsection{Chapitres}

Chaque chapitre (chacun couvrant une section de l'interface ou une section de tutoriel) doit être écrit dans un fichier séparé dans le dossier "chapitres".
Les chapitres sont inclus dans le fichier principal, "Cecilia5\_manuel.tex". Ex.:

\begin{verbatim}
\chapter{Template}
\input{chapitres/template}
\end{verbatim}

\noindent
Les tags :

\begin{verbatim}
\section{titre}
\subsection{titre}
\subsubsection{titre}
\end{verbatim}

\noindent
seront utilisés pour délimiter les sections et créer la table des matières.

\subsection{Images}

Les images, en format PDF, doivent résider dans un dossier, portant le nom du chapitre, à l'intérieur du dossier "images".
Une image simple, centrée, est insérée à l'aide de la commande "insertImage", qui prend 3 paramètres, soit:

\begin{enumerate}
    \item La taille de l'image est pouce.
    \item Le path du fichier, à partir du dossier "images".
    \item La légende à afficher.
\end{enumerate}

\noindent
Exemple d'image insérée à l'aide de la ligne suivante:

\begin{verbatim}
\insertImage{3}{template/preferences}{Exemple d'insertion d'image simple.}
\end{verbatim}

\insertImage{3}{template/preferences}{Exemple d'insertion d'image simple.}

\insertImageLeft{2.5}{template/postprocessing}{
La section Post-Processing permet d'appliquer des effets à la suite du traitement produit par le module courant.
\bigskip
\bigskip
\bigskip
\bigskip
\bigskip
}

\insertImageRight{2.5}{template/postprocessing}{
La section Post-Processing permet d'appliquer des effets à la suite du traitement produit par le module courant.
\bigskip
\bigskip
\bigskip
\bigskip
\bigskip
}

\end{verbatim}

\noindent
Les tags :

\begin{verbatim}
\section{titre}
\subsection{titre}
\subsubsection{titre}
\end{verbatim}

\noindent
seront utilisés pour délimiter les sections et créer la table des matières.

\subsection{Images}

Les images, en format PDF, doivent résider dans un dossier, portant le nom du chapitre, à l'intérieur du dossier "images".
Une image simple, centrée, est insérée à l'aide de la commande "insertImage", qui prend 3 paramètres, soit:

\begin{enumerate}
    \item La taille de l'image est pouce.
    \item Le path du fichier, à partir du dossier "images".
    \item La légende à afficher.
\end{enumerate}

\noindent
Exemple d'image insérée à l'aide de la ligne suivante:

\begin{verbatim}
\insertImage{3}{template/preferences}{Exemple d'insertion d'image simple.}
\end{verbatim}

\insertImage{3}{template/preferences}{Exemple d'insertion d'image simple.}

\insertImageLeft{2.5}{template/postprocessing}{
La section Post-Processing permet d'appliquer des effets à la suite du traitement produit par le module courant.
\bigskip
\bigskip
\bigskip
\bigskip
\bigskip
}

\insertImageRight{2.5}{template/postprocessing}{
La section Post-Processing permet d'appliquer des effets à la suite du traitement produit par le module courant.
\bigskip
\bigskip
\bigskip
\bigskip
\bigskip
}

\end{verbatim}

\noindent
Les tags :

\begin{verbatim}
\section{titre}
\subsection{titre}
\subsubsection{titre}
\end{verbatim}

\noindent
seront utilisés pour délimiter les sections et créer la table des matières.

\subsection{Images}

Les images, en format PDF, doivent résider dans un dossier, portant le nom du chapitre, à l'intérieur du dossier "images".
Une image simple, centrée, est insérée à l'aide de la commande "insertImage", qui prend 3 paramètres, soit:

\begin{enumerate}
    \item La taille de l'image est pouce.
    \item Le path du fichier, à partir du dossier "images".
    \item La légende à afficher.
\end{enumerate}

\noindent
Exemple d'image insérée à l'aide de la ligne suivante:

\begin{verbatim}
\insertImage{3}{template/preferences}{Exemple d'insertion d'image simple.}
\end{verbatim}

\insertImage{3}{template/preferences}{Exemple d'insertion d'image simple.}

\insertImageLeft{2.5}{template/postprocessing}{
La section Post-Processing permet d'appliquer des effets à la suite du traitement produit par le module courant.
\bigskip
\bigskip
\bigskip
\bigskip
\bigskip
}

\insertImageRight{2.5}{template/postprocessing}{
La section Post-Processing permet d'appliquer des effets à la suite du traitement produit par le module courant.
\bigskip
\bigskip
\bigskip
\bigskip
\bigskip
}

\end{verbatim}

\noindent
Les tags :

\begin{verbatim}
\section{titre}
\subsection{titre}
\subsubsection{titre}
\end{verbatim}

\noindent
seront utilisés pour délimiter les sections et créer la table des matières.

\subsection{Images}

Les images, en format PDF, doivent résider dans un dossier, portant le nom du chapitre, à l'intérieur du dossier "images".
Une image simple, centrée, est insérée à l'aide de la commande "insertImage", qui prend 3 paramètres, soit:

\begin{enumerate}
    \item La taille de l'image est pouce.
    \item Le path du fichier, à partir du dossier "images".
    \item La légende à afficher.
\end{enumerate}

\noindent
Exemple d'image insérée à l'aide de la ligne suivante:

\begin{verbatim}
\insertImage{3}{template/preferences}{Exemple d'insertion d'image simple.}
\end{verbatim}

\insertImage{3}{template/preferences}{Exemple d'insertion d'image simple.}

\insertImageLeft{2.5}{template/postprocessing}{
La section Post-Processing permet d'appliquer des effets à la suite du traitement produit par le module courant.
\bigskip
\bigskip
\bigskip
\bigskip
\bigskip
}

\insertImageRight{2.5}{template/postprocessing}{
La section Post-Processing permet d'appliquer des effets à la suite du traitement produit par le module courant.
\bigskip
\bigskip
\bigskip
\bigskip
\bigskip
}
